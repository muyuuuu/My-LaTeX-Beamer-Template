\documentclass[no-math]{ctexbeamer}
\usepackage{amsmath}
\usepackage{graphicx}
\usepackage[numbers,sort&compress]{natbib}
\usepackage{unicode-math}

\usefonttheme{serif}
\bibliographystyle{unsrtnat}
\setCJKmainfont{SourceHanSansCN-Normal}
\setcounter{tocdepth}{1}
\setmainfont{texgyrepagella.otf}[
    Extension      = .otf,
    UprightFont    = *-regular,
    BoldFont       = *-bold,
    ItalicFont     = *-italic,
    BoldItalicFont = *-bolditalic,
]

\mode<presentation>
{
  \usetheme{Madrid}       % or try default, Darmstadt, Warsaw, ...
  \usecolortheme{default} % or try albatross, beaver, crane, ...
  \setbeamertemplate{caption}[numbered]
} 

\title[Problem Background]{图划分报告——以问题背景为主}
\author{兰铃}
\institute{1307}
\date{\today}

\begin{document}

\begin{frame}
  \titlepage
\end{frame}

\begin{frame}{Outline}
  \tableofcontents
\end{frame}

\section{静态图划分}

\begin{frame}{静态图划分BGPLP}
    \begin{block}{问题背景与约束}
        \begin{itemize}
            \item 各个子图负载要求均衡;
            \item 各个子图之间的联系尽可能少。
        \end{itemize}
    \end{block}
    论文的处理流程:\citet{target_distribution_1}
    \begin{itemize}
        \item hash分配节点的标签;
        \item 为顶点选择标签,达到两个约束限制,设计的公式为$\text{score}(v,l)$和$bestLabel(v)$;
        \item 双收敛:边割率小于$\epsilon$或负载的平衡率小于$\lambda$。
    \end{itemize}

    \begin{alertblock}{疑问}
        当多次迭代后算法仍然不能取得较大优化效果,那么就有理由相信算法已经达到收敛状态。
    \end{alertblock}
\end{frame}

\bibliography{book}

\end{document}